\section{Introduction to the function Pointers:}


\section{qsort}
\begin{itemize}
    \item \textbf{Setter (Set the Value)} and \textbf{Getters (Get the Values) Methods}: These are essential OOP concepts for setting and retrieving an object's attribute values.
    \item The program illustrates OOP principles in C, a language without native OOP support, by utilizing structures and function pointers.
    \item A structure named `Person` is defined to represent an object with an `age` attribute, akin to a class in OOP languages.
    \item Function pointers `set` and `get` are incorporated into the `Person` structure to simulate methods for setting and getting the `age`.
    \item Two functions, `setAge` and `getAge`, are defined to manipulate the `age` attribute of a `Person` object.
    \item In the main function, an instance of the `Person` structure is instantiated, and the function pointers are assigned to point to `setAge` and `getAge`.
    \item The program employs these function pointers to set and retrieve the age of a `Person` object, exemplifying encapsulation by concealing the internal state from external access.
\end{itemize}
\break

\begin{lstlisting}
#include <stdio.h>

 struct Person {
         int age;
         // In structures we can not define functions
         // But we can Define variables
         // If we can define variable, so we can define pointer variables too
         // and if we can define pointer variables, we can define pointer functions too
         void (*set)(struct Person *, int);
         int (*get)(struct Person *);
 };


 void setAge(struct Person * instance, int age){
         instance -> age = age;
 };

 int getAge(struct Person * instance){
         return(instance-> age);
 };

 /* Creating Objects: */
 int main(){
         struct Person person1;
         // Binding:
         person1.set = setAge;
         person1.get = getAge;

         //Setting the age for the Object person1 to 18:
         person1.set(\&person1, 18);
         printf("The age is: %d \n", person1.get(\&person1));

         return(0);
 }

\end{lstlisting}
