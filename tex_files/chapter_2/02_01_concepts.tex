A variable has 5 main properties:
\begin{itemize}
    \item variable type (in this example variable type is: int)
    \item variable name (in this example variable name is: variable)
    \item variable value (in this example variable size is: 20)
    \item variable location in the memory (in this example variable location is: 0x7ffdd65ea944)
    \item number of bytes, that it has occupied int the memory (in this example variable occupies: 4 bytes)
\end{itemize}

\begin{lstlisting} 
#include <stdio.h>

int main(){
        
        int variable = 20;

        printf("This variable value is: %d  \n ", variable);
        printf("This variable occupies %lu bytes  \n", sizeof( int ) );
        printf("This variable Address  in memory is: %p  \n", &variable);
       
        return(0);
 }
\end{lstlisting}


Compile Result:
\begin{lstlisting} 
This variable value is: 20
This variable occupies 4 bytes
This variable Address  in memory is: 0x7ffdd65ea944
\end{lstlisting}

Keynotes about the Pointers:
\begin{itemize}
    \item A pointer is a variable that stores the memory address of another variable as its value.
    \item A pointer is a variable, so its value can be changed too
    \item A pointer can manipulate the address, which is holding.
    \item Pointer like other variables needing a datatype, and variable name, which is prefixed with an asterisk `*`
    \item Pointer like other variables should initialized and then used.
    \item Pointers are assigned the address of another variable, which has an \underline{same data type}.
    \item ampersand operator (\&) fetches variable's address.
    \item Pointer \underline{with} * represent the \underline{Data} stored at that memory location \textbf{(Line 12)}
    \item Pointer \underline{without} * represent the \underline{memory address}, in which the data is stored on.\textbf{(Line 9)}
\end{itemize}

