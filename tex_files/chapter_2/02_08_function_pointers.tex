function pointer will save the memory address of an function
Code:
\begin{lstlisting} 
 #include <stdio.h>

 void function(int x)
 {
         printf("X: %d \n", x);
 }

 int main(){

         void (*function_pointer)(int);

         function_pointer = &function;

         (*function_pointer)(4);

         return(0);
 }
\end{lstlisting}

Compile Results:
\begin{lstlisting} 
X: 4
\end{lstlisting}


\begin{lstlisting} 
 #include <stdio.h>

 double add(double x, double y){
         return x + y;
 }

 int main(){

         double (*add_function_ptr) (double, double);

         add_function_ptr = add;

         double result = add_function_ptr(20, 10);

         printf("result: %f \n", result);

         return 0;
 }
\end{lstlisting}

\subsection{Array of pointers to functions}

\begin{lstlisting} 
 #include <stdio.h>

 double add(int x, int y){
         return x + y;
 }

 double subtract(int x, int y){
         return x - y;
 }

 double multiply(int x, int y){
         return x * y;
 }

 double divide(int x, int y){
         return x / y;
 }


 int main(){
         double (*opration_array[]) (int, int) = {add, subtract, multiply, divide};

         double result = (*opration_array[2])(3, 20);

         printf("result: %f \n", result);

 }
\end{lstlisting}

\begin{lstlisting} 
result: 60.000000
\end{lstlisting}

\subsection{A Function That Return a Pointer To a Function:}
\begin{lstlisting} 
#include <stdio.h>

 double add(int x, int y){
         return x + y;
 }

 double subtract(int x, int y){
         return x - y;
 }

 double multiply(int x, int y){
         return x * y;
 }

 double divide(int x, int y){
         return x / y;
 }


 void ( *select_operation() ) {
         int option = 0;
         printf("Select an operation: \n");
         printf("Enter 1 for Subtracting\n");
         printf("Enter 2 for adding\n");
         printf("Enter 3 for multipling\n");
         printf("Enter 4 for dividing\n");

         scanf("%d", &option);

         switch (option){
                 case 1: return subtract;
                 case 2: return add;
                 case 3: return multiply;
                 case 4: return divide;
                 default: return NULL;

         }

 }

 int main(){

         double (*operation) (int, int) = select_operation();
         printf("result: %f \n", operation(20, 5));
 }
\end{lstlisting}
Compile Result:
\begin{lstlisting} 
Select an operation:
Enter 1 for Subtracting
Enter 2 for adding
Enter 3 for multipling
Enter 4 for dividing
3
result: 100.000000
\end{lstlisting}

\subsection{A Function That Get a Pointer To a Function As an Input:}
int this example we have an function(`is\_forzen`), which will check wether some input temperature
will cause frozen water or not, but we don't know exactly wether the temperature is in
Celsius or Fahrenheit, so this function will accept a pointer to a function.
parameter `temperature\_check` is a pointer to a function which accept an integer
as an argument and return true or false, thus we can use multiple function to checkout whether
is it frozen or not. As its illustrated in example code below two function are used. one of them
called `temperature\_check\_in\_celsius` and other one is `temperature\_check\_in\_fahrenheit`.  
In other word, we can change the behavior of `is\_forzen` function by changing the function, that we are 
passing in to it.Depending on function the program will check is it frozen based on a Celsius or
Fahrenheit. 
\newline

functions like `temperature\_check` are known as \textbf{callback functions}, why they will be used or called later in the 
body of other functions to perform some operations.


\begin{lstlisting} 
 #include <stdio.h>
 #include <stdbool.h>

 bool temperature_check_in_celsius(int temperature){
         if (temperature <= 0) return true;
         else return false;
 }


 bool temperature_check_in_fahrenheit(int temperature){
         if (temperature <= 32) return true;
         else return false;
 }


 void is_forzen ( bool (*temperature_check) (int) ){
         int temperature;
         printf("Enter the temperature: ");
         scanf("%d", &temperature);

         if(temperature_check(temperature)){
                 printf("It's frozen");
         }
         else{
                 printf("It's NOT frozen! \n");
         }
 }


 int main(){
         printf("Celsius \n");
         is_forzen(temperature_check_in_celsius);

         printf("Fahrenheit \n");
         is_forzen(temperature_check_in_fahrenheit);

 }
\end{lstlisting}


Compile Result:
\begin{lstlisting} 
Celsius
Enter the temperature: 10
Its NOT frozen!
Fahrenheit
Enter the temperature: 10
Its frozen
\end{lstlisting}

⚠  Notes:
\begin{enumerate}
        \item Space in memory is not allocated for function pointers because they are pointers to instructions and not memory address.
        \item Pointer arithmetic is not done with function pointers.
\end{enumerate}

