\subsection{Manual Assignment}


\begin{lstlisting}
 #include <stdio.h>
 #include <stdlib.h>
 #include <string.h>

 struct person{
         char name[50];
         int age;
 };

 struct person *create_person(const char *name, int age){
         struct person *p = malloc( sizeof(struct person) );
         if ( p == NULL ) {
                 return NULL;
         }

         strncpy(p->name, name, sizeof(p->name) - 1 );
         p->name[sizeof(p->name) - 1] = '\0'; //Makeing sure that string ends up with null
         p->age = age;

         return p;

 }

 int main() {
         struct person *person_instance = create_person("Ramtin", 22);

         if ( person_instance != NULL ){
                 printf("Person created: %s, %d years old", person_instance->name, person_instance->age);
                 free(person_instance);
         }

         return 0;
 }

\end{lstlisting}

Compile Result:
\begin{lstlisting}
Person created: Ramtin, 22 years old
\end{lstlisting}




create\_person function is intended to create and return a pointer to a struct person.
\newline
\begin{verbatim}
    struct person *create_person(const char *name, int age){
        ...
    }
\end{verbatim}
The create\_person function is intended to create and initialize a new struct person instance.
Here is why such a function is useful:


\begin{enumerate}
\item \textbf{Encapsulation: } By using a function to create the structure, you encapsulate the creation logic. This makes the code cleaner and easier to maintain.
\item \textbf{Initialization: } The function can initialize the structure\textquotesingle s members (like n\_item, start, end, and the data array)
 to appropriate values, ensuring the structure is in a valid state when it\textquotesingle s created.
\item \textbf{Memory Allocation: } If the data array needs to be dynamically allocated based on the value of n, the function can handle this allocation. This is especially important for the flexible array member data[].
\end{enumerate}


\subsection{Using memset}



\subsection{Designated Initializers}
code:
\begin{lstlisting}
 #include <stdio.h>
 #include <stdlib.h>
 #include <string.h>

 struct Date {
         int month;
         int year;
 };

 struct person {
         char name[50];
         int age;
         int id;
         struct Date birthdate;
 };

 struct person *create_person(const char *name, int age, int id, int birthdate_month, int birthdate_year){
         struct person *p = malloc( sizeof(struct person) );
         if (p == NULL){
                 return NULL;
         }

         *p = (struct person) {
                 .age           = age,
                 .id            = id,
                 .birthdate     = {.month = birthdate_month, .year = birthdate_year}

         };

         strncpy(p->name, name, sizeof(p->name) - 1 );
         p -> name[sizeof(p->name) - 1] = '\0'; //null-termination

         return p;

 }

 int main() {
     struct person *p = create_person("Alice", 30, 1, 5, 1990);
     if (p != NULL) {
         printf("Person created: %s, %d years old, ID: %d, Birthdate: %d/%d\n", p->name, p->age, p->id, p->birthdate.month, p->birthdate.year);
         free(p);
     }
     return 0;
 }               
\end{lstlisting}

This part of the code is called Designated Initialization:
\begin{verbatim}
    *p = (struct person) {
        .age           = age,
        .id            = id,
        .birthdate     = {.month = birthdate_month, .year = birthdate_year}
    
    };
\end{verbatim}

Compile Result:
\begin{lstlisting}
Person created: Alice, 30 years old, ID: 1, Birthdate: 5/1990
\end{lstlisting}
