\documentclass{article}
\usepackage[utf8]{inputenc}

%programming code package
\usepackage{listings}
\usepackage{color}

\definecolor{dkgreen}{rgb}{0,0.6,0}
\definecolor{gray}{rgb}{0.5,0.5,0.5}
\definecolor{mauve}{rgb}{0.58,0,0.82}

\lstset{frame=tb,
  language=C,
  aboveskip=3mm,
  belowskip=3mm,
  showstringspaces=false,
  columns=flexible,
  basicstyle={\small\ttfamily},
  numbers=none,
  numberstyle=\tiny\color{gray},
  keywordstyle=\color{blue},
  commentstyle=\color{dkgreen},
  stringstyle=\color{mauve},
  breaklines=true,
  breakatwhitespace=true,
  tabsize=3
}


\title{C Data Structures, Pointers, and File Systems}
\author{Ramtin Behesht Aeen}
\date{June 2024}

\begin{document}


\maketitle

\section{Arrays}
\subsection{Concepts}

In C: Array is a collection of consecutive objects with same data type
\begin{itemize}
    \item Array is a variable
    \item Array has a data type and name with square bracket
    \item Within the brackets are the number of elements in the array

    \begin{lstlisting}
        
    float best_score[3] = {
        1.1, 2.1, 3.1, 4.1
    };

    \end{lstlisting}

\end{itemize}    
   
% \begin{itemize}    
%     \item NoSuchFieldException : It is thrown when a class does not contain the field (or variable) specified.
    
    
%     \item ClassNotFoundException :This Exception is raised when we try to access a class whose definition is not found
    
%     \item NoSearchMethodException : It is thrown when accessing a method which is not found.
% \end{itemize}


\subsection{Working With Arrays}
\subsection{Passing an array to an function}
\subsection{Multi-Dimensional Array}
\section{Structure}
\subsection{Concepts}
\subsection{Nesting an Structure}
\subsection{Array of Structure}
\subsection{Sending a Structure to a function}
\section{Union}


\end{document}
